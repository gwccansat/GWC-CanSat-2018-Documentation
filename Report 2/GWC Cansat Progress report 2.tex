\documentclass{report}
\usepackage{graphicx} %Images and figures
\usepackage{titlepic} %Logo in title
\usepackage[margin=1in]{geometry} %margin control
\usepackage{hyperref} %hyperlinks


\title{\bf Cansat UK Progress Report 2}
\author{
	GWC Cansat\\
	George Watson's College	
}
\titlepic{\includegraphics[width=8cm]{Logo}}
\date{December 2017}


\begin{document}
\maketitle
\tableofcontents
\pagenumbering{arabic}

\chapter{Progress Report}
	\section{Progress Statement}%We need to shorten this
	We have continued along our current plan, working on each 
	section of the project simultaneously in sub-teams. A summary 
	of our progress is shown below:

	\begin{description}
		\item[Team Administration] \hfill \begin{itemize}
			\item A member of our team has dropped out due to personal
			reasons. In order to maintain our forward momentum, we
			have taken on an additional member.
			\item We have recieved three new sponsorships; two have 
			provided us with extra funding surrounding our project,
			the third company providing us with free custom PCB 
			manufacturing.
			\item Following our success with using Slack messenger for 
			internal communication and meeting scheduling,
			we are continuing with using this, further improving this
			by integrating our team's GitHub repositories with the relevant
			channels.
			\item Having sections of our team that have responsibilites for
			different	areas of the cansat development was leading to issues
			with team members not knowing how the overall	project was 
			progressing. The team leaders now collaborate to produce an
			internal report of progress every 2-3 weeks. This has recieved
			positive feedback within the team.
			\item We are continuing to utilise the sponsored licesnces we 
			recieved from TeamGantt to produce weekly gantt charts
			to ensure that we stay on schedule within each sub-team, in
			addition to an overall layout of our progress.
			\item We are continuing to maintain our website, hosted on the
			domain \url{gwccansat.com}, in addition to having launched our
			video blog (`vlog') programme, showing our progress on a weekly
			basis, leading up to the competition.
		\end{itemize}
		
		\item[Mechanical development] \hfill \begin{itemize}
			\item We continue to work towards having a finalised 
			CAD model. This is dependent upon space requirements of
			our sensors and custom PCBs, so is still undergoing iterative
			revisions.
			\item We have established ideal locations for the communications 
			and GPS antennae, learning from the issues we had in last year's
			European competition. This involves having an exterior 
			enclosure that will vastly improve communication quality and
			accuracy of GPS lock.
			\item Additionally, we are still working towards having a modular
			compartment for our secondary mission, allowing the cansat to
			be easily reused for other experiments with minimal physical
			changes made --- a simple case of plugging in and programming
			the new, relevant sensors.
		\end{itemize}

		\item[Electronics design] \hfill \begin{itemize}
			\item Our team hit some budgeting issues, leading us to
			consider other electronics options --- primarily in--house 
			assembly, as an alternative to PCB fabrication. This was recently
			resolved with the addition of our new sponsors, providing us
			with free PCB fabrication and more funding.
			\item We now have a definitive bill of materials (BOM) for 
			electronic components and materials that we will need.
			\item We also now have an initial schematic design, that is 
			ready to be made into a PCB design, provided that the software
			team have successfully interfaced and tested all the sensors and
			the mechanical team have provided ideal locations for the larger
			components on the PCB.
		\end{itemize}

		\item[Software design] \hfill \begin{itemize}
			\item Currently, the project progress is fullcruming off the
			dependency on testing the sensors from our current design.
			Hence, we have allocated extra team members and 
			resources to this section.
			\item We are currently working on testing each individual sensor, 
			to ensure that we are able to interface with them correctly with
			the current wiring solution and ready to make notes of any required 
			ammendments.
			\item Additionally, we are able to use elements of last year's code
			to assist us when creating the final cansat program and base
			station program.
		\end{itemize}

		\item[Secondary mission] \hfill \begin{itemize}
			\item We continue to strive towards having a diamagnetically 
			stabilised levitation exeriment to harvest vibration energy. We have
			unfortunately run into several technical problems, so are currently
			still working on having a functional experiment that produces power.
			\item As a precaution for the secondary mission not working, we 
			have begun brainstorming several other experiments to put in the
			secondary mission module. This is made easier by our modular
			experimental compartment in our cansat design.
			\item We have developed a plan to have multiple experiments ready
			by the time of the competition, so that we have a backup options
			in case of failure to prepare the main experiment. It will also allow
			us to demonstrate the concept of switching the expermental module 
			quickly.
		\end{itemize}

	\end{description}
		

\end{document}